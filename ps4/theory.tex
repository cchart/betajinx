\documentclass[english]{article}
\usepackage[T1]{fontenc}
\usepackage[latin9]{inputenc}
\usepackage{babel}
\begin{document}

\title{EECS 476 Mobile Robotics Problem Set 4 Theory of Operation}

\author{Steven Crilley, Charles Hart, Kan Jia, Kenneth Watka}

\maketitle
Our theory of operation for the handling of unexpected halting of
robot motion is defined by two operating modes: normal operation mode
and emergency stop mode. In normal operation mode, the robot continuously
integrates inferred differential pose information to create an estimate
of trajectory execution. This information is inferred from idealized
estimates of progress in two-dimentional path space, where each path
segment has a length and a type (turn in place, arc about a radius,
and straight line). For each path segment a maximum velocity is specified,
and a trapezoidal envelope is applied to the velocity curve, limiting
acceleration to not exceed pre-specified values. 
\end{document}
